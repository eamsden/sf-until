\documentclass{article}

\title{Deriving Correct and Efficient FRP Implementations from Categorical Denotations}
\author{Edward Amsden}
\date{\today}

\begin{document}
\maketitle

\section{Introduction}

Functional Reactive Programming (FRP) is a promising concept which enables the compositional and equational construction and analysis of reactive systems. Recent work has applied basic concepts from category theory to show a Curry-Howard correspondence between Linear Temporal Logic and Functional Reactive Programming. The categorical equations used in these demonstrations capture precise notions of the modality of propositions in LTL and of types in FRP.

FRP has been said to have ``resisted efficient implementation~\cite{Elliot}.'' In particular, most extant FRP implementations inhabit a spectrum in which pure semantic correctness and efficiency of implementation are opposed, and very few inhabit the parts of the spectrum nearest semantic correctness.

The dichotomy may be falsified by demonstrating a way to derive efficient implementations rigorously from the semantic definitions of FRP. Therefore, once the semantics for FRP have been reviewed (Section~\ref{section:semantics}), a procedure for deriving implementations from the semantics is given (Section~\ref{section:derivation}), shown to produce efficient implemementations and justify optimizations (Section~\ref{section:efficiency}) and shown to be correct (Section~\ref{section:correctness}).

\section{Review of Categorical Semantics for FRP}
\label{section:semantics}
There have been several presentations of categorical semantics for FRP~\cite{Jeffrey,Krishnaswami}. The clearest such explication~\cite{Jeltsch} denotes FRP types with objects in an exponential category from a simple category representing time to a Cartesian Closed Category with Coproduts (CCCC) representing the underlying functional language semantics. Objects from the CCCC can be lifted pointwise to objects (functors) in the exponential category. We can then define more interesting functors using these pointwise functors and products and coproducts from the CCCC. 


\end{document}
